%%%%%%%%%%%%%%%%%%%%%%%%%%%%%% Preamble
\documentclass[11pt]{article}
\setlength{\parskip}{\baselineskip}%
\setlength{\parindent}{0pt}%
\usepackage{amsmath,amssymb,amsthm,physics,graphicx,titling,hyperref}
\usepackage[margin=0.45in]{geometry}
\usepackage[export]{adjustbox}
\usepackage{color}
\usepackage{caption}
\usepackage{verbatim}
\usepackage{fancyvrb}
\captionsetup{font=footnotesize}

\newcommand{\subtitle}[1]{%
  \posttitle{%
    \par\end{center}
    \begin{center}\large#1\end{center}
    \vskip0.5em}%
}

\usepackage{graphicx}
\begin{document}

%%%%%%%%%%%%%%%%%%%%%%%%%%%%%% Heading
	\title{Ph 21 - Assignment 6 - Principal Componenent Analysis}
	\author{Yovan Badal}
	\date{06/09/2018}
	\maketitle
	
%%%%%%%%%%%%%%%%%%%%%%%%%%%%%% Body
\section{2D Linear dataset}
We simulate a simple 2D array of linearly dependent datapoints, including errors (simulated by multiplying the samples by samples from a normal distribution with mean 1). Testing our PCA analysis code gives the following output:
\verbatiminput{linear_test.txt}

We can easily observe that our PCA analysis correctly identifies a single degree of freedom in our data, since only one of the components contributes significantly to the covariance.

\section{4D Linear dataset}
We simulate a simple 4D array of linearly dependent datapoints by a simple extension of the code for the 2D linear dataset, including error. Testing our PCA analysis code gives the following output:
\verbatiminput{linear4d_test.txt}

Again we observe that our PCA analysis correctly identifies a single degree of freedom in our data, since only one of the components contributes significantly to the covariance.
\end{document}